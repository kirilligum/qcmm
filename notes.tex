%%fakesection
\documentclass{article}
\usepackage[usenames,dvipsnames]{color}
\usepackage{soul}
\usepackage{fullpage}
\usepackage{xcolor}
\usepackage{hyperref}
\usepackage{pifont}


\let\Item\item
\renewcommand\item{\normalcolor\Item}

\newcommand\ml{\color[RGB]{153, 150, 204}} %may be later
\newcommand\later{\color[RGB]{153, 204, 150}} %later
\newcommand\nn{\color[RGB]{124, 124, 255}} %no need to do
\newcommand\done{\color[RGB]{129, 180, 185} \ding{52} }
\newcommand\now{\color[RGB]{255, 0, 0}} %current

\begin{document}



\section{keys}
\begin{itemize}
  \item \ml may be later
  \item \done done
  \item \nn no need to do
  \item \now current 
\end{itemize}

t[...] for tags

t[priorety unsorted] or t[pu]-- not sorted in priorety yet

\section{doing}
\subsection{add bath}
created on Fri Jan  3 21:07:00 PST 2014
\label{sub:add_bath}
follow steve's code to add bath
% subsection add_bath (end)
\section{to do}
\subsection{ask steve about coupling}
\label{sub:ask_steve_about_coupling}
why there is no \verb`2*v12*sqrt((n1+e.n_shift)*(n2+e.n_shift))*(xn1*xn2+pn1*pn2);`
% subsection ask_steve_about_coupling (end)
\subsection{if n<=0 ask steve}
Thu Jan  2 18:41:20 PST 2014
\label{sub:if_n_0_ask_steve}
in the file harmonic.h++, why is there a line \verb`if( n <= 0 ) return 0;`
% subsection if_n_0_ask_steve (end)
\section{to do priority unsorted}
\section{plan}
\section{past}
\subsection{plot n1 and r}
done: Mon Jan  6 11:48:28 PST 2014
created on Fri Jan  3 21:04:31 PST 2014
\label{sub:plot_n1_and_r}
make a loop over integrate and \verb`push_back()` the result to a vector\\
took \verb`ode_ste` from the previous code and used \verb`partial_sum` to integrate\\
\begin{itemize}
  \item \done output the whole state vector that contains r
  \item \done output time, n1, n2 and energy to another file
\end{itemize}
% subsection plot_n1_and_r (end)
\subsection{hamiltonian}
Fri Jan  3 20:46:55 PST 2014
Thu Jan  2 20:00:06 PST 2014
\label{sub:hamiltonian}

write a hamiltonian of the pcet system then do numerical derivatives then analytical

% subsection hamiltonian (end)
\subsection{aa to cart for electron}
Thu Jan  2 19:22:14 PST 2014
Thu Jan  2 19:22:25 PST 2014
\label{sub:aa_to_cart_for_electron}

finished. in my previous code, there was a mistake x and p were calculaed in opposit but it shouldn't be a problem.

the mathematica file has formulas for the conversion. 

would be nice to have a better refference for the conversion

% subsection aa_to_cart_for_electron (end)
\section{notes}
\end{document}
